\documentclass{article}
\usepackage{graphicx} % Required for inserting images
\usepackage{mdframed}
\usepackage{amsmath}
\usepackage{amssymb}
\usepackage[textheight = 24cm]{geometry} 
\usepackage[font = sf, justification=raggedright]{caption} 
\usepackage{array}
\usepackage{hhline} 
\usepackage[math]{cellspace}
\cellspacetoplimit 4pt
\cellspacebottomlimit 4pt

% Commands new
\newmdtheoremenv{theo}{Theorem}



\title{Nahuel Notes}
\author{Alejandro Ceccheto}
\date{February 2024}

\begin{document}
	
	\maketitle
	
	
	\pagebreak
	\tableofcontents
	\pagebreak
	
	
	\section{Números Reales y sus operaciones}
	\begin{theo}
		Los números reales comprende todos los números que pueden llegar a aparecer dentro de una recta real $\mathbb{R}$, sean enteros $\mathbb{Z}$, naturales $\mathbb{N}$, racionales $\mathbb{Q}$ e irracionales $\mathbb{I}$
	\end{theo}
	
	\vspace{1cm}
	
	Example : 
	
	$\mathbb{N} = 1, 2, 3, \dots$ 
	
	$\mathbb{Q} = \frac{1}{2}, 0.4, \dots$ 
	
	$\mathbb{I} = \sqrt{2}, \sqrt{7}, \pi, e, \dots$ 
	
	$\mathbb{Z} = -536, -345, -1, -2, \dots$ 
	
	En resumen, los numeros $\mathbb{R}$ comprendes todos los numeros que existen siempre y cuando no nos metamos en los numero complejos $\mathbb{C}$.
	
	
	\subsection{Reglas de Fracciones}
	
	\subsubsection{Suma de Fracciones}
	\begin{equation*}
		\frac{a}{b} + \frac{c}{d} = \frac{ad+bc}{bd}
	\end{equation*}
	\subsubsection{Resta de fracciones
	\begin{equation*}
		\frac{a}{b} - \frac{d}{c} = \frac{ad-bc}{bd}
	\end{equation*}
	\subsection{Reglas de Exponentes}
	
	
	
	\subsection{Reglas de radicales}
	
	
	
	\subsection{Intervalos}
	
	
	\subsection{Números con coma a fracción}
	
	
	\subsection{Pitagóricas en recta real}
	
	
	
	\section{Errores, relativos, porcentuales y absolutos}
	\subsection{Error Relativo}
	
	\subsection{Error Absoluto}
	
	\subsection{Error Porcentual \% }
	
	
	\section{Funciones}
	\subsection{Función Cuadrática}
	
	
	\subsection{Función Modulo}
	
	
	\subsection{Función Polinómica}
	
	
	\subsection{Función Exponencial y Logarítmica}
	
	
	\subsection{Función Homogénea}
	
	
	\section{Números Complejos}
	
	\subsection{Operaciones con números complejos}
	
	\subsection{Operaciones combinadas con números complejos}
	
	
	
	\begin{theo}
		
	\end{theo}
\end{document}