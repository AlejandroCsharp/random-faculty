\documentclass{article}
\usepackage{pst-eucl}
\usepackage{changepage, amsmath, amssymb, pgfplots, tikz}
\usetikzlibrary{fit,positioning}
\usepgfplotslibrary{groupplots}
\usepackage[utf8]{inputenc}
\psset{PointName=none,PointSymbol=none}
\usetikzlibrary{arrows.meta}	
\usetikzlibrary{calc,patterns,angles,quotes}
\pgfplotsset{compat=1.16}
\usetikzlibrary{arrows.meta}
\usepackage{fullpage}
\usepackage{xparse,array}
\usetikzlibrary{circuits.ee.IEC}
\usepackage{ulem}
\usepackage{geometry,graphicx}				% Paquetes adicionales.
\usepackage{subcaption,array}
\usepackage{multicol,multirow}
\usepackage{mathtools}
\usepackage{circuitikz,siunitx}
\usepackage{xcolor}
\usetikzlibrary{circuits.ee.IEC}
\usepackage[colorlinks=true, urlcolor=mypurple, linkcolor=mypurple]{hyperref}
\usetikzlibrary{trees}
\usepackage[dvipsnames]{xcolor}
\usepackage{darkmode}
\enabledarkmode
\usepackage{framed} % or, "mdframed"
\usepackage{fancyhdr}
\usepackage{fontawesome}
\pagestyle{fancy}
\usepackage[framed]{ntheorem}
\usepackage{lmodern}
\usepackage{tabularx}
\usepackage{microtype}
\setlength{\columnsep}{1.5cm}
\setlength{\columnseprule}{0.4pt}
\usepackage{multicol}
\usetikzlibrary{arrows, shapes.gates.logic.US, calc}
\usetikzlibrary{calc,arrows.meta}
\usepackage[most]{tcolorbox}
\definecolor{p1}{HTML}{caf0f8}
\definecolor{p2}{HTML}{ade8f4}
\definecolor{p3}{HTML}{90e0ef}
\definecolor{p4}{HTML}{48cae4}
\definecolor{p5}{HTML}{00b4d8}
\definecolor{p6}{HTML}{0096c7}
\definecolor{p7}{HTML}{0077b6}
\definecolor{p8}{HTML}{023e8a}
\definecolor{p9}{HTML}{03045e}
\usepackage{multicol}
\usepackage{colortbl}
\colorlet{xcol}{blue!60!black}
\usepackage{capt-of}
\definecolor{myred}{HTML}{f44336}
\definecolor{mypink}{HTML}{e81e63}
\definecolor{mypurple}{HTML}{9c27b0}
\definecolor{mydeeppurple}{HTML}{673ab7}
\definecolor{myindigo}{HTML}{3f51b5}
\definecolor{myblue}{HTML}{2196f3}
\definecolor{mylightblue}{HTML}{03a9f4}
\definecolor{mycyan}{HTML}{00bcd4}
\definecolor{myteal}{HTML}{009688}
\definecolor{mygreen}{HTML}{4caf50}
\definecolor{mylightgreen}{HTML}{8bc34a}
\definecolor{mylime}{HTML}{cddc39}
\definecolor{myyellow}{HTML}{ffeb3b}
\definecolor{myamber}{HTML}{ffc107}
\definecolor{myorange}{HTML}{ff9800}
\definecolor{mydeeporange}{HTML}{ff5722}
\definecolor{mybrown}{HTML}{795548}
\definecolor{mygray}{HTML}{9e9e9e}
\definecolor{mybluegray}{HTML}{607d8b}
\definecolor{pag}{HTML}{293133}
\usepackage[labelformat=empty]{caption}
\usepackage{mdframed}
\newtcolorbox{qq}[2][]{%
	boxrule=0.75pt,
	sharp corners,  % Square edges
	colframe=white,  % Set the color of the outline
	colback=pag,  % Set the color of the fill
	coltext=white,
	#1,
}
\newframedtheorem{frm-thm}{Theorem}

\usepackage{tikz-3dplot}
\usetikzlibrary{3d,backgrounds,intersections}
% small fix for canvas is xy plane at z % https://tex.stackexchange.com/a/48776/121799
\makeatletter
\tikzoption{canvas is xy plane at z}[]{%
	\def\tikz@plane@origin{\pgfpointxyz{0}{0}{#1}}%
	\def\tikz@plane@x{\pgfpointxyz{1}{0}{#1}}%
	\def\tikz@plane@y{\pgfpointxyz{0}{1}{#1}}%
	\tikz@canvas@is@plane}
\makeatother

\usepgfplotslibrary{groupplots}

\tikzset{
	conditional line/.style 2 args={
		/utils/exec={\pgfmathsetmacro{\xcoordA}{#1}},
		/utils/exec={\pgfmathsetmacro{\xcoordB}{#2}},
		insert path={
			\ifdim\xcoordA pt>\xcoordB pt
			(\xcoordA,0) -- (\xcoordB,0)
			\fi
		},
	},
}

\newcommand{\Logo}{
	\begin{tikzpicture}[remember picture,overlay]
		\node at ($(current page.north east) + (-1cm,-1cm)$) {AC};
	\end{tikzpicture}
}




%--------------------------------------------------------------------



\newcommand{\framedtext}[1]{%
	\par%
	\noindent\fbox{%
		\parbox{\dimexpr\linewidth-2\fboxsep-2\fboxrule}{#1}%
	}%
}


\tikzset{
	venn box/.style={
		draw=black, very thick, 
		rounded corners=10,
		inner xsep=10pt, inner ysep=15pt, outer ysep=5pt
	},
	venn numbers/.style={
		%    draw,
		inner ysep=0pt,
		align=center
	},
	venn title/.style={
		fill=black, text=white
	}
}


\ExplSyntaxOn
\NewDocumentCommand{\truthtable}{mm}
{
	\group_begin:
	\setlength{\arraycolsep}{0pt}
	\crocket_truth_make_preamble:n { #1 }
	\crocket_truth_make_body:n { #2 }
	\crocket_truth_make:
	\group_end:
}

% variables
\tl_new:N \l__crocket_truth_preamble_tl
\tl_new:N \l__crocket_truth_first_tl
\tl_new:N \l__crocket_truth_body_tl
\seq_new:N \l__crocket_truth_rows_seq

% internal functions
\cs_new_protected:Nn \crocket_truth_make_preamble:n
{
	\tl_clear:N \l__crocket_truth_preamble_tl
	\tl_clear:N \l__crocket_truth_first_tl
	% examine the first argument in order to build the table
	% preamble and the first row
	\tl_map_inline:nn { #1 }
	{
		\tl_if_in:VnTF \c_crocket_truth_delims_tl { ##1 }
		{% a delimiter is inserted in @{...} as a phantom (almost)
			% and not taken into account for the first row
			\tl_put_right:Nn \l__crocket_truth_preamble_tl { @{ \__crocket_truth_delim:n {##1} } }
		}
		{
			\tl_if_eq:nnTF { | } { ##1 }
			{% vertical bar is ignored in the rows, but it gives a | in
				% the table preamble (actually, quad | quad)
				\tl_put_right:Nn \l__crocket_truth_preamble_tl { @{\hspace{1em}}|@{\hspace{1em}} }
			}
			{% otherwise, c is added to the table preamble and
				% the item to the first row
				\tl_put_right:Nn \l__crocket_truth_preamble_tl { c }
				\tl_put_right:Nn \l__crocket_truth_first_tl { ##1 & }
			}
		}
	}
	% since rows will end with &, we add a dummy trailing column
	\tl_put_right:Nn \l__crocket_truth_preamble_tl { c } % empty last column
}

\cs_new_protected:Nn \crocket_truth_make_body:n
{% start building the table body
	\tl_clear:N \l__crocket_truth_body_tl
	% split the argument at \\
	\seq_set_split:Nnn \l__crocket_truth_rows_seq { \\ } { #1 }
	% and map it
	\seq_map_inline:Nn \l__crocket_truth_rows_seq
	{% each item is a row; map it to add &
		\tl_map_inline:nn { ##1 }
		{% but ignoring |
			\tl_if_eq:nnF { | } { ####1 }
			{
				\tl_put_right:Nn \l__crocket_truth_body_tl { \crocket_truth_tf:n { ####1 } & }
			}
		}
		% append the row terminator
		\tl_put_right:Nn \l__crocket_truth_body_tl { \\ }
	}
}

\cs_new_protected:Nn \crocket_truth_make:
{% make the table
	\exp_args:NnV \begin {array} \l__crocket_truth_preamble_tl
	% temporary define \__crocket_truth_delim:n to produce its argument
	\noalign { \cs_gset_eq:NN \__crocket_truth_delim:n \use:n }
	% output the first row
	\tl_use:N \l__crocket_truth_first_tl \\
	\hline
	% make \__crocket_truth_delim:n to deliver a phantom
	\noalign { \cs_gset_eq:NN \__crocket_truth_delim:n \hphantom }
	\tl_use:N \l__crocket_truth_body_tl
\end{array}
}

\cs_new_protected:Nn \crocket_truth_tf:n { \makebox[1.2em]{ $ \mathrm { #1 }$ } }

\tl_const:Nn \c_crocket_truth_delims_tl { ()[]\{\} }

\ExplSyntaxOff


%--------------------------------------------------------------------
\newcommand{\bs}{\textcolor{bluefill}{'}}
\definecolor{bluefill}{RGB}{41, 49, 51}
\renewcommand\columnseprulecolor{\hspace{50pt}}

% Portrail


\title{Algebra I}
\date{\today}
\author{Alejandro Ceccheto}



\begin{document}
	%% School Portrail.
	\maketitle
	\pagebreak
	
	

	\begin{frm-thm}[Proposiciones Lógicas]
		Si un número es par entonces es múltiplo de 3 \\
		\begin{center}
			x par $\Rightarrow$ x mult. de 3
		\end{center}
	\end{frm-thm}


	
	
	
\begin{multicols}{3}
	\begin{minipage}[t]{0.5\textwidth}
		\fbox{p, q, r} = Proposiciones. \\
		
		\begin{itemize}
			\item 1, 0 $\rightarrow$ prop. \\
			
			\item 1 y 0 $\rightarrow$ p $\land$ q\\
			
			\item 1 o 0 $\rightarrow$ p $\lor$ q \\
			
			\item o 1 o 0 $\rightarrow$ p $\veebar$ q \\
			
			\item si 1 entonces 0 $\rightarrow$ p $\Longrightarrow$ q \\
			
			\item equivalencia $\rightarrow$ p $\Longleftrightarrow$ q \\
			
			\item Negación $\rightarrow$  $>$p, $\vec{p}$, $\sim$p, -p \\
		\end{itemize}
	\end{minipage}
	
	Negación =
	\begin{minipage}{0.5cm}
		\[
		\truthtable{
			P | \sim P
		}{
			V | F \\
			F | V
		}
		\]
	\end{minipage}
	
	Conjunción =
	\begin{minipage}[c]{0.5cm}
		\[
		\truthtable{
			P Q | P \land Q
		}{  V V | \bs V \\
		    V F | \bs F \\
		    F V | \bs F \\
		    F F | \bs F \\
		}
		\]
	\end{minipage}
	
	
    Disyunción = 
	\begin{minipage}[c]{0.5cm}
		\[
		\truthtable{
			P Q | P \lor Q
		}{  V V | \bs V \\
		    F V | \bs V \\
		    V F | \bs V \\
		    F F | \bs F \\
		}
		\]
		
	\end{minipage}
	                                                                                          
	
	Condicional =
	\begin{minipage}[c]{1cm}
		\[
		\truthtable{ P Q | P \Rightarrow Q
		}{           V V | \bs V \\
		             V F | \bs F \\
		             F V | \bs V \\
		             F F | \bs F \\
		}
		\]
	\end{minipage}
	
	
	Bicondicional= 
	\begin{minipage}[c]{1cm}
		\[
		\truthtable{P Q	| P \Leftrightarrow Q 
		}{ V V | \bs V \\
		   V F | \bs F \\
		   F V | \bs F \\
		   F F | \bs V \\
		}
		\]
	\end{minipage}
\end{multicols}

Entonces, con estas lógicas algebraicas podemos deducir 3 cosas, si son Tautología, que significa que siempre va a ser verdadero, Contradicción, que significa que son siempre falsas, o Contingencia $\rightarrow$ a veces V, otras F.

\begin{multicols}{2}
	\begin{minipage}{0.5\textwidth}
		\[
		\truthtable{
			P Q | r | P \lor Q | P \lor Q \Rightarrow r  
		}{  V V V \bs V \bs \bs V \bs \bs \\
			V V F \bs V \bs \bs F \bs \bs \\
			V F V \bs F \bs \bs V \bs \bs \\
			V F F \bs F \bs \bs V \bs \bs \\
			F V V \bs F \bs \bs V \bs \bs \\
			F V F \bs F \bs \bs V \bs \bs \\
			F F V \bs F \bs \bs V \bs \bs \\
			F F F \bs F \bs \bs V
		}
		\]
	\end{minipage}
	
	\begin{tikzpicture}
	\node {$\sim$p$\lor$q}
	[edge from parent fork left,grow=left]
	child {node {q}}
	child {node {$\sim$p}
		child {node {p}}
	};
	\node [below right, text width=10cm,align=justify] at (1,-0.5)
	{ (por cultura general)
	};
\end{tikzpicture}
\end{multicols}
\pagebreak

\vspace{-1.2cm}
\section{Ejericios hechos.}

\vspace{0.5cm}


\begin{itemize}
	\item[1. ] $\sim$(p$\lor$q)

\[
\truthtable{
	P Q | P \lor Q | \sim (p\lor q)
}{  
	V V | \bs V \bs\bs F \\
	F V | \bs V \bs\bs F \\
	V F | \bs V \bs\bs F \\
	F F | \bs F \bs\bs V 
}
\]



\item[2. ]P$\Rightarrow$(q$\land$$\sim$q)
\[
\truthtable{
	P Q | \sim Q | (Q \land \sim P) | P \Rightarrow (Q \land \sim Q)
}{  
	V V | \bs F \bs F \bs\bs\bs\bs F \\
	V F | \bs V \bs F \bs\bs\bs\bs F \\
	F V | \bs F \bs F \bs\bs\bs\bs V \\
	F F | \bs V \bs F \bs\bs\bs\bs V
}
\]

\item[3. ] p $\Rightarrow$ (q $\lor$ $\sim$ q)
\[
\truthtable{
	P Q | \sim Q | (Q \lor \sim Q) | P \Rightarrow (Q \lor \sim Q)
}{  V V | \bs F \bs  V \bs  \bs  \bs \bs V \\
    V F | \bs V \bs  V \bs  \bs  \bs \bs V \\
    F V | \bs F \bs  V \bs  \bs  \bs \bs V \\
    F F | \bs V \bs  V \bs  \bs  \bs \bs V \\
} 
\]
\end{itemize}


Para mas referencias y ejemplos visitar\footnote{Uno de los libros que mas se asemeja a la forma de explicar de la profe. \\
\url{https://archive.org/details/AlgebraIArmandoRojo/page/n13/mode/2up}}.

  




\end{document}